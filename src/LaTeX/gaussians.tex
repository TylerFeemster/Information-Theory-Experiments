\documentclass[12pt]{article}
 
\usepackage[margin=.75in]{geometry} 
\usepackage{amsmath, tikz, enumitem, amsthm, amscd, amssymb, graphicx, multicol, array, mathtools, verbatim}
\usetikzlibrary{decorations.markings}
\DeclarePairedDelimiter{\abs}{\lvert}{\rvert}

\newcommand{\Q}{\mathbb{Q}}
\newcommand{\N}{\mathbb{N}}
\newcommand{\Z}{\mathbb{Z}}
\newcommand{\R}{\mathbb{R}}
\newcommand{\C}{\mathbb{C}}
\newcommand{\F}{\mathbb{F}}
\newcommand{\m}{\lambda}
\newcommand{\e}{\epsilon}
\newcommand{\al}{\alpha}
\newcommand{\be}{\beta}
\newcommand{\om}{\omega}
\newcommand{\GL}{\text{GL}}
\newcommand{\chr}{\text{char }}
\newcommand{\im}{\text{im}}
\newcommand{\Aut}{\text{Aut }}
\newcommand{\id}{\text{id}}

\newtheorem{remark}{Remark}
\newtheorem{proposition}{Proposition}
\newtheorem{theorem}{Theorem}
\newtheorem{lemma}{Lemma}

\title{Gaussians}
\author{Tyler Feemster}
\date{\today}

\begin{document}

\maketitle

The entropy of a probability distribution $P$ over $\R$ is given by
$$H(X) = -\int_{\R}P(x)\log P(x) dx$$ 

If $X$ is the standard normal distribution, then
$$H(X) = -\int \frac{1}{\sqrt{2\pi}}e^{-x^2/2}\log\left(\frac{1}{\sqrt{2\pi}}e^{-x^2/2}\right)dx$$
$$= -\frac{1}{\sqrt{2\pi}}\int e^{-x^2/2}\left(-\frac{1}{2}\log 2\pi - \frac{x^2}{2}\right)dx$$
$$= \frac{\log 2\pi}{2} + \frac{1}{2\sqrt{2\pi}}\int x^2 e^{-x^2/2}dx.$$

Isolating the integral and using integration by parts, we have
$$\int x^2 e^{-x^2/2}dx = \int x (x e^{-x^2/2})dx = -\int 1\cdot (-e^{-x^2/2})dx = \int e^{-x^2/2} dx = \sqrt{2\pi}.$$
Thus,
$$H(X) = \frac{1}{2}(1+\log 2\pi)$$

\section*{Kernel Formula}

For $n$ a non-negative integer, define 
$$I(n) = \int_{-\infty}^{\infty}x^n e^{-x^2/2} dx.$$
Using polar coordinates, we can obtain the popular result that
$$I(0) = \sqrt{2\pi}.$$
For $n$ odd, the integrand is odd, so $I(n) = 0$.

Now, we can use integration by parts to obtain
$$I(2k) = \int x^{2k} e^{-x^2/2} dx = \int x^{2k-1}(xe^{-x^2/2}) dx$$
$$= - \int (2k-1)x^{2k-2}(-e^{-x^2/2})dx = (2k-1)\int x^{2(k-1)}e^{-x^2/2}dx = (2k-1)I(2k-2).$$

By induction, we can see that
$$I(2k) = \frac{(2k)!\sqrt{2\pi}}{2^k k!}.$$


\end{document}